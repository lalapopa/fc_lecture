\documentclass{article}

\usepackage{import}
\usepackage{graphicx}
\usepackage{color}
\usepackage[dvipsnames]{xcolor}
\usepackage[warn]{mathtext}
\usepackage[utf8]{inputenc}
\usepackage[T2A]{fontenc}
\usepackage[english,russian]{babel}
\usepackage{upgreek}

\usepackage{wrapfig}
\usepackage{amsmath}
\usepackage{tipa}
\usepackage{tikz}
\usepackage[
  top=2cm,
  bottom=2cm,
  left=3cm,
  right=2cm,
]{geometry}
\usetikzlibrary{positioning}
\usetikzlibrary{shapes.geometric}
\usetikzlibrary{fit}
\usetikzlibrary{calc,patterns,angles,quotes}
\usetikzlibrary{babel}
\usetikzlibrary{decorations.markings}
\usepackage{pgfplots}
\usepgfplotslibrary{groupplots}
\usepgfplotslibrary{external} 
\tikzexternalize



%Header
\usepackage{fancyhdr}
\pagestyle{fancy}
\fancyhf{}

\renewcommand{\sectionmark}[1]{\markboth{#1}{}}
\fancyhead[L]{\leftmark}
\fancyhead[R]{Управление движением ЛА}
\fancyfoot[C]{\thepage}
\setcounter{secnumdepth}{0} %delete number before section



\usepackage{mathtools, nccmath} % break eqation in two line 
\usepackage{float}



\begin{document}
\thispagestyle{empty}
\tableofcontents
\newpage

\section{Лекция 1}
\subsection{Общие положения}
\textbf{Цель курса:}
Дать необходимые знания для понимания принципов построения и работы современных систем ручного и автоматического управления самолета, а так же умение применять мат. методы синтеза и анадиза этих систем.\\
Любой полет можно рассматривать как последовательность выполнения следующих режимов (этапов полета):
\begin{enumerate}
  \item Взлет.
  \item Набор высоты с разгоном.
  \item Крейсерский полет.
  \item Смена эшелона.
  \item Стабилизация заданной линии пути в горизонтальной плоскости.
  \item Смена линий пути.
  \item Разгон и торможенние на постоянной высоте.
  \item Снижение с торможением.
  \item Заход на посадку.
  \item Полет по глисаде и посада.
\end{enumerate}

Каждый этап полета характерезуется определенным законом изменения параметров полета. Например: На режими крейсерского полета, $H=const$, $V=const$, выполнение этого условия достигается при определенном законе отклонения органов управления, в качестве которых выступают: руль высоты, элероны, руль направления, а так же РУД (регулирует тягу).
\begin{flushright} 
\textit{
По этому управление ЛА состоит в формировании отклонения органов управления для требуемого изменения параметров движения ЛА в условиях действия разлчного рода возмущений.}
\end{flushright}
При автоматическом управлении закон отклонения органов управления формируется автоматикой без участия летчика.

При полуавтоматическом управлении (ручном) в формировании закона управления учавствует летчик.

При ручном управлении отклонения рулей в продольном и поперечном канале осуществляется летчиком с помощью штурвального управления через отклонение центральной ручки управления (РУС) или штурвала.

Отклонение органов управления в путевом канале осуществляется с помощью педялей.

Система управления полетом — это коплекс технических устройств преобразующих управляющее воздействие летчика и командные сигналы автоматических устройств в отклонение органов управления. 

В настоящее время система управления самолетом включает в себя:
\begin{enumerate}
  \item Система штурвального управления (СШУ) --- она обеспечивает отклонение орагнов управления в соответствии с отклонением летчиком рычагов управления и сигналами системы устойчивости и управляемости самолёта (СУУ).
  \item Система автоматического управления (САУ) --- применяется для стабилизации параметров движения и их движения по заданным алгоритмам.
  \item Система управления механизацией крыла --- используется на взлетно-посадочных режимах.
  \item Система управления тягой двигателя.
  \item Система управления передней стойкой шасси.
\end{enumerate}

Работа указанных систем обеспечивается системой энергопитания, которая отбирает часть мощности двигателя. СШУ с точки зрения выполняемых функциональных задач и обеспечения безопасности полета является основной системой. Проблема создания системы управления является важнейшим аспектом в проектировании самолета. СШУ существенным образом влияет на важнейшие характеристики пилотирования самолёта: устойчивость, управляемость, оказывает влияние на формирование облика самолёта ЛТХ. СШУ должна обеспечивать:
\begin{enumerate}
  \item Максимальное использование маневренности самолета.
  \item Наипростейшее пилотирование на всех этапах полёта.
  \item Ограничение выхода самолёта на предельные режимы.
\end{enumerate}

САУ предназначена для автоматизации полета по типовой траектории (от взлёта до посадки) с целью снижения загрузки экипажа особенно при длительном пилотировании. Повышение точности выполнения режимов полёта, а также снижение погодного минимума и повышение регулярности полётов. Тенденция расширения функций автоматического управления обусловленна снижением числа ошибок, происходящих по вине лётчика статистика показывает что 80\% катостроф происходит по вине лётчика. Автоматизация посадки в условиях плохой видимости компенсация воздействия отказов требующих быстрой ответной реакции позволяет существенно уменьшить число ошибок которые совершаются по вине лётчика.

Таким образом первая часть курса посвещена изучению СШУ, синтезу автоматических устройств (демпферы, автоматы устойчивости) составляющих СУУ.

Необходиме знания:
\begin{itemize}
  \item Динамика полета
  \item ТАУ
\end{itemize}
\newpage

\section{Лекция 2}
\subsection{Комплекс оборудований систем самолета}

Автоматическое и ручное управление самолетом осуществляется с помощью разнообразных устройств и приборов, которые можно разбить на две функциональные части:
\begin{itemize}
\item{Система СШУ}
\item{Пилотажный навигационный комплекс(ПНК)}
\end{itemize}

Навигационная система(НС): Бортовая цифровая вычислительная машина (БС), Инерциальная система (ИС), Датчики системы стабилизации (ДС), РСДМБ/РСБМ --- радиостанции ближнего и дальнего наведения. \\
Измерители: Гировертикаль, курсовая система(КС), система воздушных сигналов (СВС).\\
Система отображения информации (СОИ): Индикация пилотажно-навигационной обстановки(ПНО), командный прибор.\\
Система автоматического управления(САУ): Система тракторного управления, система взлета и посадки, автопилот, автомат тяги.\\
Система штуравльного управления(СШУ): Исполнительная часть системы управления(Привода), автоматические устройства.

\begin{figure}[h]
\begin{tikzpicture}
\node [draw,
    fill=Goldenrod,
    minimum width=0.8cm,
    minimum height=0.8cm,
]  (bs) at (0,0){БС};

\node [draw,
    fill=Goldenrod,
    minimum width=0.8cm,
    minimum height=0.8cm,
    below=0.25cm of bs,
]  (ds) {ДС};

\node [draw,
    fill=Goldenrod,
    minimum width=0.8cm,
    minimum height=0.8cm,
    below=0.25cm of ds,
]  (is) {ИС};

\node [draw,
    fill=Goldenrod,
    minimum width=0.8cm,
    minimum height=0.8cm,
    right=0.1cm of bs,
    text width=1cm
]  (rsdm) {РСДМ РСБМ};

\node [fit=(bs)(ds)(is)(rsdm),draw,dashed,black](ns) {};
\node [above=0.1cm of ns](textns){НС};



\node [draw,
    fill=Goldenrod,
    minimum width=0.8cm,
    minimum height=0.8cm,
    right=2.5cm of bs,
]  (ind) {Индикация ПНО};

\node [draw,
    fill=Goldenrod,
    minimum width=0.8cm,
    minimum height=0.8cm,
    below=0.25cm of ind,
]  (cp) {Командный прибор};

\node [fit=(cp)(ind),draw,dashed,black](soi) {};
\node [above=0.1cm of soi](textsoi){СОИ};

\node [draw,
    fill=Goldenrod,
    minimum width=0.8cm,
    minimum height=0.8cm,
    below=1.5cm of is,
]  (gv) {Гировертикаль};

\node [draw,
    fill=Goldenrod,
    minimum width=0.8cm,
    minimum height=0.8cm,
    below=0.25cm of gv,
]  (svs) {СВС};

\node [draw,
    fill=Goldenrod,
    minimum width=0.8cm,
    minimum height=0.8cm,
    below=0.25cm of svs,
]  (ks) {КС};

\node [fit=(svs)(gv)(ks),draw,dashed,black](izm) {};
\node [above=0.1cm of izm]{Измерители};

\node [draw,
    fill=Goldenrod,
    minimum width=0.8cm,
    minimum height=0.8cm,
    right=1cm of gv,
]  (tru) {Траекторное управление};

\node [draw,
    fill=Goldenrod,
    minimum width=0.8cm,
    minimum height=0.8cm,
    below=0.25cm of tru,
]  (tals) {TALS};

\node [draw,
    fill=Goldenrod,
    minimum width=0.8cm,
    minimum height=0.8cm,
    below=0.25cm of tals,
]  (ap) {A/P};

\node [draw,
    fill=Goldenrod,
    minimum width=0.8cm,
    minimum height=0.8cm,
    below=0.25cm of ap,
]  (at) {A/T};

\node [fit=(tru)(tals)(ap)(at),draw,dashed,black](sau) {};
\node [above=0.1cm of sau]{САУ};
\node [fit=(sau)(izm)(textns)(textsoi),draw,dashed,black](pnk) {};
\node [above=0.1cm of pnk]{ПНК};
\node [draw,
    fill=Goldenrod,
    minimum width=0.8cm,
    minimum height=0.8cm,
]  (pi) [right=3cm of pnk, yshift=3cm]{Летчик};

\node [draw,
    fill=Goldenrod,
    minimum width=0.8cm,
    minimum height=0.8cm,
    below=2.5cm of pi,
]  (isp) {Исполнительная часть};

\node [draw,
    fill=Goldenrod,
    minimum width=0.8cm,
    minimum height=0.8cm,
    below=1cm of isp,
]  (avt) {Автоматические устройства};
\node [fit=(isp)(avt),draw,dashed,black](schu) {};
\node [above=0.1cm of schu]{СШУ};

\node [draw,
    fill=Goldenrod,
    minimum width=0.8cm,
    minimum height=0.8cm,
    below=1cm of avt,
]  (org) {Органы управленя};

\node [left=0.5cm of sau.north](con1){};
\node [left=1.2cm of izm.north](con2){};
\node [left=1cm of isp.north](con3){};
\node [below=0.2cm of ind.west](con4){};


\draw [-latex] (ns.east) --++(0.3,0) |- (con4.center);
\draw [-latex] (izm.east) --++(0.5,0) |- (ind.west);
\draw [-latex] (izm.east) --++(0.4,0) |- (sau.west);
\draw [-latex] (ns.south) --++(0,-0.25) -| (con1.center);
\draw [-latex] (con2.center) --++(0,+0.8) -| (ns.south);
\draw [-latex] (pnk.east) --++ (1,0) |- (pi.west);
\draw [-latex] (pi.south) --++ (0,-1) --++ (-1,0) -|(con3.center);
\draw [-latex] (sau.east) --++ (1,0) |- (isp.west);
\draw [-latex] (izm.south) --++ (0,-3) --++(7.75,0)|- (avt.west);
\draw [-latex] (avt.north) -- (isp.south);
\draw [-latex] (isp.east) --++ (1,0) --++ (0,-2.5) -| (org.north);
\end{tikzpicture}
\end{figure}
СШУ --- совокупность всех средств передающих управляющее воздействие летчика и САУ на органы управления самолётом, а также автоматические устройства улучшения пилотажных и летных характеристик самолета, которые работают параллельно с лётчиком. Является основой системы управления, так как она непосредственно взаимодействует с органами управления, через СШУ отрабатываются сигналы САУ. СШУ должна обеспечивать:
\begin{enumerate}
\item Максимальное использование маневренности самолёта.
\item Наибольшую простоту пилотирования на всех режимах. 
\item Ограничение выхода самолёта на предельные режимы полёта
\end{enumerate}
ПНК --- совокупность измерителей индикаторов и автоматических устройств с помощью которых на борту создается информационная модель полёта и результаты задачи стабилизации, автоматического и полуавтоматического управления. ПНК должна обеспечивать: 
\begin{enumerate}
\item Стабилизацию режимов полётов.
\item Навигацию и автоматическое управление траекторией по программе.
\item Автоматический взлёт и посадку.
\item Определение пилотажно навигационных параметров и их отображение лётчику.
\end{enumerate}
\subsection{Способы управления самолётом}
Для современных систем управления характерна функциональная избыточность, проявляющаяся в том, что управление самолетом может осуществлятся несколькими способами: 
\begin{enumerate}
\item Автоматическое управление траекторией --- все режимы програмируются в вычислителе навигационного комплекса управленеи самолётом производится по отклонениям от программы полёта. На основе этих отклонений вырабатываются команды на изменение углового положения, которые отрабатываются соответствующими системами стабилизации. При автоматическом управлении полётом лётчик выполняет функцию включения тех или иных систем и общего контроля за работой системы. 
\item Управление траекторным движением по командному прибору (директорное управление) --- в этом режиме управление самолётом производится лётчиком вручную через воздействие на рычаги управления. Стратегия управления --- требуемую угловую ориентацию самолёта вырабатывает автоматические системы траекторного управления, однако приведение самолёта в заданное угловое положение (отработку заданных $\vartheta$, $\theta$, $\psi$) осуществляет лётчик, а не система стабилизации углового положения. Управление производится с помощью командного прибора. На стрелки прибора выводится рассогласование между требуемым углом и текущим положением задача лётчика путем воздействия на рычаги управления устранить это рассогласование, тем самым лётчик выполняет функцию стабилизации углового положения.
\item Автоматическая стабилизация параметров движения в этом случае лётчик ориентируется по пилотажным командным приборам (не директорный) и вызывает тот или иной режим стабилизации, выполняемый автопилотом. Управление производится с помощью манипуляции с задающими устройствами. В процессе полёта лётчик может неоднократно переходить от автоматического управления к ручному и наоборот, этот способ управления называется совмещенный
\item Ручное управление по пилотажно-навигационным индикаторам или ориентирам лётчик управляет самолётом с помощью рычагов управления вклад автоматики в процесс управления минимален (выполняет функцию системы улучшения устойчивости и управляемости)
\end{enumerate}

Проблема рационального сочетания ручного управления и автоматического заключается в использовании совмещенного управления в котором обеспечивается: 
\begin{enumerate}
\item Автоматическое управление на всех режмах полёта.
\item Поканальный переход с режима автоматического управления на ручное при вмешательстве лётчика в управление.
\item Востановление режима автоматического управления после окончания вмешательства лётчика.
\item Сохранение для лётчика динамического стереотипа управления.
\end{enumerate}
В настоящее время на большинстве самолётах реализован режим совмещенного управления угловым движением самолёта, в этом режиме осуществляется стабилизация того углового положения которое имело место в момент окончения вмешательства лётчика в управление. 
Для этого необходимо:
\begin{enumerate}
\item Определение момента начала и окончания вмешательства лётчика в управление.
\item Отключить режим САУ с целью обеспечения лётчику на время вмешательства традиционного стереотипа управления.
\item По окончанию вмешательства задейстовать режим автоматической стабилизации.  
\end{enumerate}
\newpage

\section{Лекция 3} 
\subsection{Типы систем штурвального упраления}
\subsubsection{Необратимое ручное управление}
НРУ —- Широко применяется на самолетах малых и средних размерах, передача управляющего сигнала к рулям осуществляется непосредственно летчиком без использования внешних источников энергии благодаря этому достигается: простота конструкции, простота обслуживания, высокая наджность. 
В систему НРУ выходят:
\begin{enumerate}
\item Рычаги управления самолетом, штурвал, педали, механическая проводка управления, тросовая или жесткая в виде тяг.
\item Органы управления $\delta_{ОУ} = K_ш X_py$
\end{enumerate}

Соответственно усилие на рычагах управления $Р_ру$ от шарнирных аэродинамических моментов определяется соотношением:
\[
P_{ру} = K_{ш} M_{ш}
\]

В общем случае уровень усилий должен быть сопоставим c возможностями летчика поэтому НРУ имеют средства снижения шарнирного момента. При НРУ сигналы САУ обрабатываются специальным приводом. связь привода САУ с проводкой осуществляется через устройство пересиливания и отключения (муфта)


\begin{figure}[h]
\import{./images/}{figure_one.pdf_tex}
\end{figure}    

\subsubsection{Необратимое бустерное управление}

\begin{figure}[h]
\import{./images/}{figure_two.pdf_tex}
\end{figure}

В системе НБУ шарнирный момент органа управления полностью воспринимается бустером. Появление НБУ обусловленно развитием авиацонной техники характерной особенностью которого является: расширение области режима полета (сверхзвук) и увеличение размеров самолета. Использование НБУ создала возможность улучшения характеристик устойчивости и управляемости с помощью специальных автоматических устройств (СУУ). Вместе с рычагами управления и механической проводкой основными элементами являются рулевой привод и загрузочное устройство с триммерным механизмом и система гидравлического питания. Все системы должны иметь высокую надежность, что достигается многократным резервированием для НБУ характерны следющие особенности:
\begin{enumerate}
\item Усилия управления в НБУ определяется характеристиками искусственной загрузки независимо от шарнирного момента усилия можно сделать такими какие можно сделать для обеспечения управляемости и безопасности.
\item В НБУ могут быть включены различные устройства(САУ, СУ). Исполнительные устройства этих систем могут быть небольшой мощности т.к они не рассчитаны на $М_ш$, а на усилие входной части привода.
\item Механическая проводка может быть облегчена.
\item На базе НБУ может быть построена система электродистнционная система управления(ЭДСУ).
\end{enumerate}
\subsubsection{Электродистнционная система управления}
\begin{figure}[h]
\import{./images/}{figure_three.pdf_tex}
\end{figure}

ЭДСУ - система в которой связь между рычагом управления и исполнительными приводами реализовано не механически, а электрическим(по проводам) это позволяет достичь высокой точности и быстродействия в отработке сигнала: летчика, САУ и СУ, благодаря отсутствию зазоров, трения, упругости механической проводки ЭДСУ просиходит постепенное вытеснение аналоговой в замен цифровой.

ЦCДУ имеет более широкие возможности в реализации сложных законов управления. А также более глубокой эффективной организации обмена данными благодаря ЦСДУ открылась возможность интеграции всех систем в единый комплекс предназначенный для решения задач пилотирования самолета. Применение ЭДСУ позволило использовать мало габаритные рычаги управления (боковая ручка) это помогает летчику управлять на больших перегрузках, улучшает обзор приборной доски, снижает вес рычагов управления. Основная проблема обеспечения надежности не ниже чем у механических систем.
\newpage

\subsubsection{Включение автоматических устройств в СШУ}
\begin{figure}[h!]
\import{./images/}{figure_four.pdf_tex}
\end{figure}

Где ДП --- дистанционны передатчик.

Под СУУ понимается автоматическая система которая обеспечивает необходимый уровень статических и динамических характеристик управляемости самолета. СУУ должна обеспечивать приемлимый стереотип пилотажных характеристик самолета который должен быть ориентировать на определение психо-физиологических возможностей летчика, как звена контура управления «Летчик -- СУ -- самолет» фактически для летчика не важно каким образом СУ обеспечивает заданные характеристики ему важно, чтобы объект управления(ОУ) обладал такими характеристиками которые делают процесс управления простым и безопасным. Восприятие пилотажных характеристик автоматизированного самолета летчиком должно быть таким же, как если бы эти характеристики обеспечивались традиционными средствами. для того чтобы работа этой системы была незаметной рычаги должны оставаться неподвижными не должны возникать «отдачи» (В этом случае СУ) включается в СШУ последовательно. В ЭДСУ сигнал вычислителя СУ поступает на вход привода с электрическим входом и никак не связан с механической проводкой и следовательно его отработка никак не сказывается на отработку рычага управления. При механической проводке СШУ подобный способ включения СУ реализуется по схеме «раздвижной тяги».
САУ включается в СШУ по параллельной схеме, в этом случае привод САУ подключается к проводке таким образом, чтобы при работе САУ вся механическая проводка при управлении так же перемещалась, как при управлении самолета летчиком. Рычаги управление в этом случае являются индикатором правильной работы САУ, за время эксплуатации самолета у летчика вырабатывается стереотип правильно работы САУ и в случае отклонения от него, летчик имеет возможность взять управление на себя путем отключения САУ.
\newpage
\section{Лекция 4}
\subsection{Состав СШУ}
СШУ состоит из различной степени сложноси, выполнающиее различные функции. Разделяют следующие системы:
\begin{itemize}
\item Механическая или электродистанционная система связывающая рычаги управления с приводами. 
\item Система создания усилия на рычагах управления --- загрузочное устройство (ЗУ).
\item Система сервоприводов и рулевых приводов.
\item Система регулирования коэффициентов передачи между рычагами и рулями ($К_ш$).
\item Система ограничения предельных режимов полета ОПР. 
\item Система управления балансировкой. 
\item Система улучшения устойчивости и управляемости (через СШУ отабатывается сигнал от САУ). 
\end{itemize}
\textbf{Загрузочные устройства или автомат регулирования загрузки (ЗУ или АРЗ)}. Бустер мешает летчику ощущать нагрузку на руль ($q$) без этого управления самолтом становится практически невозможным. Следовательн  о появляются ЗУ и триммеры 
ЗУ предназначенна для обеспечения приемлемых характеристик управления. По прикрепленным к рычагам усилиям на всех режимах полета. Важным параметрами являются градиенты усилия на рычаге, градиант усилия на РУ по нормальной перегрузке $P^{n_y}$ и в боковом канале по скорости крена $P^{\omega_x}$. Также контролируется $X^{n_y}; \, X^{\omega_x}$ параметры. Для легких самолетов:
\[
P^{n_y} =10 \dots 30 \frac{Н}{ед. перегрузки}; \, X^{n_y} \ge 12 \frac{мм}{ед. перегрузки} 
\]
Можно определить требования загрузочного устройства:
\[
P^{n_y} = \frac{\Delta P}{\Delta n_y} = \frac{\Delta P}{\Delta x} \frac{\Delta x}{\Delta n_y} = P^{x} X^{n_y}
\]
\[
P^{x} = \frac{P^{n_y}}{X^{n_y}} \approx 2 \frac{H}{мм} \text{ --- характеристики загрузочного устройства}
\]

Наиболее простое устройство пружина.
Характеристиками пружины ЗУ являются предварительный натяг и излом. $P_0$ - вводится для центровки, а излом обусловлен тем, что на больших приборных скоростях, где требуются малые отклонения рулей необходимо затежеление загрузки для предотвращения раскачки самолета, а на малых скоростях, где требуются большие отклонения загрузка не должна быть чрезмерной. Существуют более сложные загрузочные устройства. Автоматически осуществляется регулирование $P^{x}$. 

Задача загрузочных устройств обеспечение примерно равное $P^{n_y}$. Усилие возникающее при отклонении рычагов летчиком снимаются с помощью механического триммера эффекта.
\begin{figure}[h!]
\centering
\import{./images/}{lec_fig_1.pdf_tex}
\end{figure}
Системы регистрируют $K_ш$ на легких самолетах с мехаической проводкой применяют автоматы двух типов:
\begin{enumerate}
\item АРЗ
\item АРУ --- который одновременно регулирует $P^{x}$ и $K_ш$ на всех режимах полета. Особенность АРЗ состоит в том, что при его работе сохраняется полный диапазон отклонения в том числе при отказе
\end{enumerate}
\[
\delta_в = K_ш X_в
\]
В случае АРУ (одновременно $P^{x}$ и $K_ш^V$) диапазон отклонения орагонов управления изменяются по режимам полета, это требует применения мер безопасности в случае отказа АРУ. 
Исполнительная часть управления полетом.
Исполнительными устройствами СУ преобразующими командные сигналы в механические перемещения органов управления явлются привода. Приводы присодены к аэро приводам называемые рулевыми. Помимо отработки командных сигналов эти выполняются училие мощности. Приводы главная задачей которго явлюятся преобразование электрического сигнала в механическое называют сервопривод. Они являются маломощными исполнительными устройствами которые обычно устанавливаются во входных части системы. В последнее время тенденция к слиянию сервоприводов с рулевыми с целью улучшения динамических свойств, улучшение надежности и уменьшение веса. Обычно называют силовыми или электроприводами. 
Любой привод представляет следующую систему: входной величиной которого являются управляющие сигналы, а выходная перемещение или скорость перемещения привода. 
\begin{figure}[!h]
\begin{tikzpicture}
\node[draw,
    fill=White,
    minimum width=3.5cm,
    minimum height=1.2cm,
    fill=White,
] (w1) at (3,0) {Сравнительное У.};

\node [draw,
    fill=White,
    minimum width=1.5cm,
    minimum height=1.2cm,
    right=1cm of w1,
]  (w2) {реш. У};

\node [draw,
    fill=White,
    minimum width=1.5cm,
    minimum height=1.2cm,
    right=1cm of w2,
]  (w3) {Силовой мех.};

\node [draw,
    fill=White,
    minimum width=1.5cm,
    minimum height=1.2cm,
    right=1cm of w3,
]  (w4) {Выходное звено};

\node [draw,
    fill=White,
    minimum width=1.5cm,
    minimum height=1.2cm,
    below=0.7cm of w2,
]  (w5) {Обратная связь};

\draw [->] (0,0) -- (w1.west) 
  node[midway,yshift = 0.2cm](text1){Упр.};

\draw [->] (w1.east) -- (w2.west);
\draw [->] (w2.east) -- (w3.west);
\draw [->] (w3.east) -- (w4.west);
\draw [->] (w4.east) --++ (1,0)
  node[midway](end_line){};


\draw [->] (end_line.center) |- (w5.east);
\node[below left=0cm of w1, xshift=0.15cm](point){};
\node[below = 0.25 of text1](){Сиг.};

\draw [->] (w5.west) --++ (-5,0) |- ([yshift = 0.5cm]point.center);
\end{tikzpicture}
\end{figure}


Если звено обратной связи несет информацию о перемещении системы, то в этом случае управляющий сигнал вызывает смещение выходного звена относительно нейстрали (привод с жесткой обратной связью). Если сигнал обратной связи пропорционален скорости изменения выходного звена, то привод со скоростной обратной связи. 
Если выход и обратная связь гидропривода являюется механической, то привод гидромеханический (бустер) при бустерном управлении летчику достаточно прикладывать небольшие усилия. 

\subsection{Ограничители предельных режимов}
\textbf{Ограничитель предельных значений $\alpha$ и $n_y$}.
Предотвращает выход самолета на предельные значения параметров движения, преследует цель: безопасность полетов, путем информирования (предупреждения) и активного воздействия на руль высоты. Для пассажирских ла наиболее активно предотвращают выход на $\alpha_пред$. Это задача решается несколькими путями световая и звуковая сигнализация, вибрация, или его дополнительная загрузка, активное ограничение путем воздействия на руль высоты. Эти способы предотвращения выхода на $\alpha_крит$ применяется с любым типом СУУ. Звуковая сигнализация и световая обезательная мера которая используется на всех пассажирских самолетах, однако в стрессовой ситуации этой меры может быть не достаточно для предотвращения возниковения опасной ситуации. Поэтому дополнительно используют более эффективный способ предупреждения --- тряска. Другим столь же эффективных способом является дополнительная загрузка рычагов управления реализуется специальным загружателем.
\[
0 - \bar{\alpha} < \alpha_кр; \, \bar{\alpha} = \alpha + K_{\dot{\alpha}} \dot{\alpha} 
\]
\[
1 - \bar{\alpha} > \alpha_кр; \, \alpha = \alpha + K_{\omega_z} \omega_z 
\]
Вторые слагаемые используются для внесения опережения, то есть учета прогноза в движении самолета по углу атаки.

\textbf{Система управления балансировкой\emph{(Рычага управления обнуление силы действующией на рычаг со стороны ЗУ. )}}
При ручном управлении балансировку нужно выполнять с помощью МТЭ по сигналам летчика МТЭ при любом положении рычага управления приводит пружину в нейтраль, пружина не сжата, не растянута при этом $P=0$ в этом случае летчик может освободить рычаг управления, рычаг остается в заданном положении, а следовательно и ОУ.
САУ отклоняет руль на $\Delta\delta_{доп}$ к баллансировочному значению $\delta_{бал}$, поэтому если летчик стримировал рычаг управления в момент когда ЛА находится на опорной траектории, то $\Delta\delta = 0$ и переход на автоматическое управление происходит плавно без рывков на органы управления.
Обратный переход от автоматического к ручному будет происходить плавно если в момент отклонения САУ рычаг управления будет находится в стриммированом состоянии --- это достигается с помощью УАТ.

\newpage
\section{Лекция 5}
\subsection{Летчик в СУ самолетом}
При пилотировании летчик осуществляет срванение фактических параметров движения самолета с заданными, оценивает величину направление и рассогласования, и с помощью рулевых органов, используя СШУ, устраняет это рассогласование. С этой точки зрения летчик и самолет являются взаимосвязанными элементами единой замкнутой СУ «Летчик --- СШУ --- Самолёт» при этом качество управления определяется как динамическими характеристиками самолета (его Устойчивость и управялемость), так и психофизиологическими особенностями летчика, а также свойствами СШУ при аналитическом исследовании этой системы удобно оперировать математическим ожиданием действий летчика, как звена этой системы. Летчик как звено СУ имеет вход и выход. Входом являются органы чувств, воспринимающие информацию по положению самолета, выходом являются мускульные усилия которые он прилагает к рычагам управления. Выход связан со входом, и центральной нервной системой внутренней и внешней обратной связью по положению и усилиям.

\begin{figure}[!h]
\begin{tikzpicture}
\node[draw,
    fill=White,
    minimum width=3.5cm,
    minimum height=1.2cm,
    fill=White,
] (w1) at (3,0) {Органы чувств};

\node [draw,
    fill=White,
    minimum width=1.5cm,
    minimum height=1.2cm,
    right=1cm of w1,
]  (w2) {ЦНС};

\node [draw,
    fill=White,
    minimum width=1.5cm,
    minimum height=1.2cm,
    right=1cm of w2,
]  (w3) {Двиг. аппарат};

\node [draw,
    fill=White,
    minimum width=1.5cm,
    minimum height=1.2cm,
    right=1cm of w3,
]  (w4) {СШУ};

\node [draw,
    fill=White,
    minimum width=1.5cm,
    minimum height=1.2cm,
    right=1cm of w4,
]  (w5) {Самолет};

\node [draw,
    fill=White,
    minimum width=1.5cm,
    minimum height=1.2cm,
    below=1.5cm of w3,
]  (w6) {ЦМР о движ.};
\node[below left=0cm of w1, xshift=0.15cm](point){};
\node[below right=0cm of w3, xshift=-0.15cm](point2){};



\draw [->] (w1.east) -- (w2.west);
\draw [->] (w2.east) -- (w3.west);
\draw [->] (w3.east) -- (w4.west);
\draw [->] ([yshift = 0.5cm]point2.center) --++ (0.5, 0) --++ (0, -1)  --++ (-10.5,0) 
node[midway, yshift=-0.25cm](){Кинестетическая ОС} |- ([yshift = 0.5cm]point.center);
\draw [->] (w4.east) -- (w5.west);
\draw [->] (w5.east) --++ (1,0)
  node[midway](end_line){};

\draw [->] (end_line.center) |- (w6.east);
\draw [->] (w6.west) --++ (-8,0) |- (w1.west);
\node [fit=(w1)(w2)(w3),draw,dashed,black](dashedbox) {};
\node [above=0.1 cm of dashedbox.north](){Летчик};

\end{tikzpicture}
\end{figure}

Внутренняя обратная связь осуществляется двигательными ощущениями называется кинестетической внешней обратной связью. Осуществляется зрением и осязанием изменение усилий, ощущается летчиком лучше нежели изменение положения РУ. Поэтому летчик управляет самолетом основываясь на восприятии усилий, это определяет использование загрузочных механизмов. Для описания динамических свойств летчика как элемента замкнутой системы используют следующую передаточную функцию:
\begin{equation} \label{W:1}
W_л = W_л^1 W_л^2 
\end{equation}
\begin{equation} \label{W:2}
 W_л = e^{-p\tau} \frac{1}{\tau_{н.м} p +1} 
\end{equation}
\begin{equation} \label{W:3}
 W_л = \frac{K_л(T_1 p + 1)}{T_2 p +1}
\end{equation}\\
ПФ (\ref{W:1}) --- отражает присущие человеку свойства временной запаздывание(задержка в реакции)\\
ПФ (\ref{W:2}) - Нервномышечное запаздывание, где $\tau = 0.13 - 0.2 \tau_н.м= 0.1c$ \\
ПФ (\ref{W:3}) - Отражает адаптивные свойства параметры изменяются в следствии адаптации, где $T_1 = 0.25 - 0.5 с$ - постоянное упреждение, $T_2 = 10 - 20 * T_1$ - характеризует динамическое запаздывание.\\
Из анализа процесса пилотирования в одноканальной задаче слежения можно сделать вывод, летчик действует так, чтобы оптимизировать суммарные характеристики замкнутой системы, с целью обеспечения минимума ошибки слежения. В связи с этим для оценки параметров летчика можно использовать правило оптимизации линейных систем.
\textbf{Задачи оптимизации.} Правила частотного анализа при этом подходе в виде требований: к полосе пропускания и запасам устойчивости, если окажется, что для выполнения этих требований потребуется значение ($К_л, \, Т_1$) лежащие за пределами человеческих возможностей, то можно сделать вывод о неблагоприятной динамики самолета, когда желаемые качества регулирования достигаются ценой психических и физиологических нагрузок.

\subsection{Устойчивость движения и управляемость самолета}
Процесс управления можно представить в виде двух задач:
\begin{enumerate}
\item Обеспечить требуемое значение перегрузок, углов атаки, скольжения, крена необходимых для реализации заданного опорного движения. В установившемся опорном движении моменты уравновешивают, а углы постоянны, однако реальное движение всегда отличается от идеального из-за внешней среды, неточности пилотирования, ветровых возмущений, пульсации силовой установки и.т.д.

\item Парирование возмущающих воздействий и сохранение заданного или близкого к нему состояния при воздействии возмущений.
\end{enumerate}
Обе задачи могут быть решены если самолет обладает управляемостью. \\
Управляемость --- способность выполнять в ответ на целенаправленные действия летчика или автоматики любой предусмотренный в процессе эксплуатации маневр. В любых допустимых условиях в том числе и при наличии возмущений. Когда возмущения малы и действуют кратковременно, то управление полетом существенно упрощается, если опорное движение устойчиво.\\
Устойчивость --- способность самостоятельно без участия летчика сохранять заданный режим полета и возвращаться к нему после непроизвольного отклонения от него под действием внешних возмущений, когда эти возмущения исчезнут.\\
Различают устойчивость к бесконечно малым возмущениям (устойчивость в «малом») и устойчивость к ограниченным конечным возмущениям(устойчивость в «большом»).
Исследование устойчивости в большом имеет смысл только тогда, когда движение устойчиво в малом. При исследовании устойчивости в малом удобно рассматривать не сами параметры возмущенного движения, а их отклонение от параметров невозмущенного движения.\\
Устойчивость и управляемость самолетом относится к числу особенно важных физических свойств самолета. От них в значительной степени зависит простота и точность пилотирования, а также полнота реализации летчиком технических возможностей самолета. Эти свойства самолета проявляются в характере протекания переходных процессов при отклонении летчиком органов управления при изменении режима работы двигателей, при воздействии атмосферных возмущений и.т.д. Очевидно предпочтительнее такие переходные процессы, которые без участия летчика приводят к быстрому восстановлению к исходному режиму полета, а при управлении приводят к его быстрой перебалансировке и не требуют от летчика чрезмерных затрат (по времени и усилиям). Наиболее желательным переходным процессом является колебательное движение с быстрым затуханием или апериодическое с малым временем переходного процесса. Недопустимо возрастающие по амплитуде колебание малого периода, и достаточно быстрое развивающееся неустойчивое апериодическое движение.\\
Устойчивость зависит от:
\begin{enumerate}
\item Фактических условий полета.
\item Особенности аэродинамической и весовой компоновки.
\item От средств автоматической стабилизации и демпфирования (СУУ).
\end{enumerate}
Управляемость зависит от:
\begin{enumerate}
\item От его устойчивости.
\item От особенностей системы (СШУ).
\item От кинематических и динамических характеристик, закона изменения усилий на рычагах управления.
\end{enumerate}

Управляемость самолета, как и устойчивость подлежат количественной оценке при полете на различных высотах и скоростях, а также при всех основных вариантах загрузки.

\newpage
\section{Лекция 6}
\subsection{Математическая модель самолета как объекта управления}
\subsubsection{Общие уравнения движения самолета}
Движение самолета как твердого тела (без учета диффамации) в спокойной атмосфере, считая постоянной массу самолета $m=const$ и не учитывая кривизну земли и гироскопического момента двигателя описываются следующей системой дифференциальных уравнений:
\[
\frac{dV}{dt} = \frac{R_{x_a}}{m} - g \sin{\theta};
\]
\[
\frac{d\theta}{dt} = \frac{1}{mV}(R_{y_a} \cos{\gamma_a} - R_{z} sin{\gamma_a} - mg \cos{\theta});
\]
\[
\frac{d\Psi}{dt} = -\frac{1}{mV \cos{\theta}} (R_{y_a} \sin{\gamma_a} + R_{z_a} \cos{\gamma_a});
\]
\[
I_x \frac{d\omega_x}{dt} - I_{xy} \frac{d\omega_y}{dt} - (I_z - I_y) \omega_z \omega_y + I_{xy} \omega_x \omega_z = M_{R_x};
\]
\[
I_y \frac{d\omega_y}{dt} - I_{xy} \frac{d\omega_x}{dt} + (I_x - I_z) \omega_x \omega_z + I_{xy} \omega_y \omega_z = M_{R_y};
\]
\[
I_z \frac{d\omega_z}{dt} + (I_y - I_x) \omega_y \omega_x + I_{xy}(\omega_y^2 - \omega_x^2) = M_{R_z};
\]
\[
\frac{d\gamma}{dt} = \omega_x - \tan{\vartheta}(\omega_y \cos{\gamma} - \omega_z \sin{\gamma});
\]
\[
\frac{d\vartheta}{dt} = \omega_z \cos{\gamma} + \omega_y \sin{\gamma};
\]
\[
\frac{d\beta}{dt} = \omega_x \sin{\alpha} + \omega_y \cos{\alpha} + \frac{1}{mV}(R_z + mg \cos{\theta} \sin{\gamma_a});
\]
\[
\frac{dL}{dt} = V\cos{\theta} \cos{\Psi};
\]
\[
\frac{dH}{dt} = V\sin{\theta};
\]
\[
\frac{dZ_g}{t} = - V \cos{\theta} \sin{\Psi};
\]

Для определения угла атаки и скоростного угла крена и рыскания используют следующие выражения:

\[
\frac{d\alpha}{dt} = \omega_z + \tan{\beta}(\omega_y \sin{\alpha} - \omega_x cos{\alpha}) - \frac{1}{mV \cos{\alpha}} (R_{y_a} - mg\cos{\theta} \cos{\gamma_a}),
\]

\[
\sin{\gamma_a} = \frac{\cos{\vartheta} \sin{\gamma} + \sin{\theta} \sin{\beta}}{\cos{\theta} \cos{\beta}}.
\]

\[
\frac{d\psi}{dt} = \frac{1}{\cos{\vartheta}}(\omega_y \cos{\gamma} - \omega_z \sin{\gamma}) 
\]

На основании этих уравнений получим упрощенное уравнение самолета для частных случаев:
уравнение продольного не возмущённого движения $OX_k Y_k Z_k$ --- в траекторной СК.

\begin{align}
\label{eqn:prodol}
\begin{split}
\dot{V}_k&= g(n_x - \sin{\theta});
\\
\dot{\theta}&= \frac{g}{V_k} (n_y - \cos{\theta});
\\
\dot{\omega_z}&= \bar{M}_{R_z};
\\
\dot{\vartheta}&= \omega_{z};
\\
\dot{H}&= V_k \sin{\theta};
\\
\dot{L}&= V_k \cos{\theta}.
\end{split}
\end{align}

Считая угол $\alpha_W$ малым то имеем:
\[
n_x = \frac{Pcos{(\alpha_k - \varphi_{дв})} - X_a}{mg}
\]

\[
n_y = \frac{P \sin{(\alpha_k + \varphi_{дв})} + Y_a}{mg}
\]
где $X_a = C_{x_a} q S$; $Y_a = C_{y_a} q S$; $\alpha_k = \vartheta - \theta$, 
$\alpha_W = \frac{W}{V_k}$ --- учитывает угол атаки за счет ветра.

$\bar{M}_{R_z} = \frac{M_{R_z}}{I_z}$; 

\[
M_{R_z} = M_z + P \bar{l}_P b_a
\]

где $M_z = m_zqS b_a$; $\bar{l}_P = \bar{x}_P \sin{\varphi_{дв}} - \bar{y}_P \cos{\varphi_{дв}}$; $\bar{y}_P$ и $\bar{x}_P$ --- координаты точки приложения тяги в свзянной СК отнесенные к $b_a$

На всех режимах управляемого полета задачей управление, является движение по заданной траектории поскольку управляющие воздействия направлены на стабилизацию программного движения, а отклонение от программной траектории вызванные возмущающими действиями малы, то для анализа этих отклонений можно использовать линеаризованную отнисительно программной трактории. Линеаризуя \eqref{eqn:prodol}, получим: 

\[
\Delta \dot{V} = g(n_x^V \Delta V - \cos{\theta_0}\Delta \theta + n_x^{\alpha_k} \Delta \alpha_k + n_x^{H} \Delta H + n_x^{\psi}\Delta \psi + n_x^{u} \delta u + n_x^{\alpha_W} \Delta \alpha_W )
\]

\[
\Delta \dot{\theta} = \frac{g}{V_0} [(n_{y}^{V} - \frac{n_{y_0} - \cos{\theta_0}}{V_0}) \Delta V + \sin{\theta_0} \Delta \theta + n_{y}^{\alpha_k} \Delta \alpha_k + n_{y}^{H} \Delta H + n_{y}^{\varphi_{дв}} \Delta \varphi_{дв} + n_{y}^{u} \Delta u + n_{y}^{\alpha_W} \Delta \alpha_{w}]
\]

\[
\Delta \dot{\omega_{z}} = \bar{M}_z^V \Delta V + \bar{M}_z^{\alpha_k} \Delta \alpha_k + \bar{M}_z^{\omega_z} \Delta \omega_z + \bar{M}_z^{\alpha_k} \Delta \dot{\alpha_k} + \bar{M}_z^H \Delta H + \bar{M}_z^\varphi \Delta \varphi + \bar{M}_z^{\alpha_{w}} \Delta \alpha_w
\]

\[
\Delta \dot{\theta} = \Delta \omega_z
\]
\[
\Delta \dot{H} = \sin{\theta_0} \Delta V + V_0 \cos{\theta_0} \Delta \theta
\]
\[
\Delta \dot{L} = \cos{\theta_0}\Delta V - V_0 \sin{\theta_0} \Delta \theta
\]
\[
\Delta \alpha_k = \Delta \vartheta - \Delta \theta
\]

Вся эта математическая модель описывает угловое и тракторное движение во взаимосвязи и ее используют когда невозможно разделить тракторное и угловое движение например в задачах посадки. На установившихся режимах можно выделить короткопериодические и длиннопериодическое при следующих допущениях: 
\begin{enumerate}
\item При описании быстро протекающего короткопериодического движения можно пренебречь изменением скорости и высоты, а также изменением режима работы двигателя. А также можно положить $g/V_0 \approx 0$
\item Длиннопериодическое движение можно рассматривать как реакцию на изменение угла тангажа и режима работы двигателя, задаваемого управляющим воздействием
\end{enumerate}
\[
\Delta V = 0, \, \Delta H = 0, \, \Delta L = 0.
\]
\[
\Delta \dot{\omega_z} = \bar{M}_z^{\alpha} \Delta \alpha_k + \bar{M}_z^{\omega_z} \Delta \omega_z + \bar{M}_z^\alpha \Delta \dot{\alpha}_k + \bar{M}_z^\phi \Delta \phi + \bar{M}_z^\alpha \Delta \alpha_k + \bar{M}_z
^{\dot{\alpha}} \Delta \dot{\alpha}_W
\]

\[
\Delta \dot{\theta} = \frac{g}{V} \Delta n_y
\]
\[
\Delta n_y = n_y^{\alpha_k} \Delta \alpha_k + n_y^{\varphi} \Delta \varphi + n_y^{\alpha_W} \Delta \alpha_W
\]
\[
\Delta \theta = \Delta \vartheta + \Delta \alpha_k
\]

Найдем передаточные функции $n_y^\phi= 0$

\begin{equation}\label{eqn:per_alpha}
W_{\frac{\alpha_k}{\varphi}}(p) = \frac{\Delta \alpha_k(p)}{\Delta \varphi(p)} = \frac{\bar{M}_z^\varphi}{p^2 + 2hp + \omega_0^2}
\end{equation}

\[
W_{\frac{\omega_z}{\varphi}} (p) = \frac{\Delta \omega_z(p)}{\Delta \varphi(p)} =\frac{\bar{M}_z^\varphi(p + \frac{g}{V} n_y^{\alpha_k})}{p^2 + 2hp + \omega_0^{2}}
\]

где $2h = \frac{g}{v}(n_y^{\alpha_k} - \bar{M}_z^{\omega_z} - \bar{M}_z^{\dot{\alpha}})$; $\omega_0^2 = - \bar{M}_z^{\alpha} - \frac{g}{V} n_y^{\alpha_k} \bar{M}_z^{\omega_z} = \frac{-qsb_a}{I_z} C_{y_a}^{\alpha} \sigma_{n}$\\
$\sigma_n = m_z^{C_y} + \frac{\bar{M}_z^{\omega_z}}{\mu}$; $\mu = \frac{2 m}{\rho S b_a}$

Перепишем в каноническом виде:
\begin{equation}\label{eqn:per_omega}
W_{\frac{\omega_z}{\varphi}} = \frac{\bar{M}_z^\varphi(p + \frac{g}{V} n_y^{\alpha_k})} {p^2 + 2hp + \omega_0^2}= %
-\frac{K_c (T_{1c} p + 1)}{T^2_c p^2 + 2 T_c \xi_c p + 1}
\end{equation}
где $T_c = \frac{1}{\sqrt{\omega_0^2}} = \sqrt{-\frac{I_z}{q S b _a \sigma_n C_{y_a}^{\alpha}}}$; $\xi_c = \frac{h}{T_c} = \frac{1}{2}( \frac{g}{V} n_y^{\alpha_k} - \bar{M}_z^{\omega_z} - \bar{M}_z^{\dot{\alpha}}) \sqrt{-\frac{-q s b_a \sigma_n C_{ya}^{\alpha}}{I_z}}$; $T_{1c} = \frac{V}{g n_y^{\alpha_k}}$; $K_c = \frac{-\bar{M}_z^\phi}{\omega_0^2 T_{1c}}$

Для самолета нормальной схемы все коэффициенты должны быть положительны в передаточных функциях \eqref{eqn:per_alpha} \eqref{eqn:per_omega}

обратимся к второму внешнему возмужению $\alpha_w$:
\[
W_{\frac{\alpha_k}{\alpha_W}} = \frac{(\bar{M}_z {\dot{\alpha} - \frac{g}{V} n_y^{\alpha_W}) p + \bar{M}_z^\alpha + \frac{g}{V} n_y^{\alpha_W} \bar{M}_z^{\omega_z} }}{p^2 + 2hp + \omega_0^2}
\]
\[
W_{\frac{\alpha}{\alpha_w}} = \frac{p(p - \bar{M}_z^{\omega_z})}{p^2 + 2hp+ \omega_0^2}
\]

\[
W_{\frac{n_y}{W_y}} = \frac{n_y^\alpha}{V} \frac{p(p - \bar{M}_z^{\omega_z})}{p^2 + 2hp + \omega_0^2}
\]
\newpage

\section{Лекция 7}
\subsection{Модель траекторного движения}
Получим уравнение траекторного движения, когда можно пренебреч действием силы на управляющей аэродинамической поверхности, выделим из системы уравений \eqref{eqn:prodol} выделим уравнение для траекторных параметров $\Delta \vartheta, \, \Delta n$, полагая $n_y^\phi = n_x^\phi = 0$ в качестве опорного движения принимаем $\theta=0$ получим СДУ в которой величины $\Delta \vartheta, \Delta n$ управляющие.


\begin{equation}
\medmath{
  \begin{multlined}
  \frac{d}{dt}
  \begin{bmatrix}
  \Delta V\\
  \Delta \Theta\\
  \Delta H\\
  \Delta L
  \end{bmatrix}
  =
  \begin{bmatrix}
  a_{11} & a_{12} & a_{13} & 0 \\
  a_{21} & a_{22} & a_{23} & 0 \\
  a_{31} & a_{32} & 0 & 0 \\
  a_{41} & a_{42} & 0 & 0 \\
  \end{bmatrix}
  \begin{bmatrix}
  \Delta V\\
  \Delta \Theta\\
  \Delta H\\
  \Delta L
  \end{bmatrix}
  + \\ 
  +
  \begin{bmatrix}
  b_{11} & b_{12}\\
  b_{21} & b_{22} \\
  0 & 0 \\
  0 & 0 \\
  \end{bmatrix}
  \begin{bmatrix}
  \Delta \vartheta\\
  \Delta n\\
  \end{bmatrix}
\end{multlined}
}
\end{equation}
где 
\[
a_{11} = g n_x^V 
\]
\[
a_{12} = -g n_x^{\alpha_к} - g \cos{\theta}
\]
\[
a_{13} = g n_x^H
\]
\[
a_{21} = \frac{g}{V} n_y ^V 
\]
\[
a_{22} = -\frac{g}{V} n_y^{\alpha_к}
\]
\[
a_{23} = \frac{g}{V} n_y^H
\]
\[
a_{31} = \sin{\theta}
\]
\[
a_{32} = V \cos{\theta}
\]
\[
a_{41} = \cos{\theta} 
\]
\[
a_{42} = -V \sin{\theta}
\]
\[
b_{11} = gn_x^{\alpha_к}
\]
\[
b_{12} = g \cos{\alpha_к + \varphi_{дв}}
\]
\[
b_{21} = -a_{22}
\]
\[
b_{22} = \frac{g}{V} \sin{\alpha_к + \varphi_{дв}}
\]

Пологая коэффициенты постоянными получаем 8 передаточных функций. Для упрощения ПФ положим $a_{11} \approx a_{21} \approx a_{13} \approx a_{23} \approx 0$ это допущение основывается на том факте, что управляющее воздействие развиваемое системой в процессе регулирования, значительно превосходит силы возникающие в следствии вариации $\Delta n$ и $\Delta \vartheta$, кроме того в получающихся выражениях $a_{42} b_{22}, \, a_{12} b_{22}$ можно пренебреч в следствии малости.
В результате в следствии вычислений получим две группы передаточных функций.
Канал тангажа:\\
\[
W_{\frac{V}{\vartheta}} = -\frac{K_V(1+T_Vp)}{p(1+T_{1c}p)}, \, W_{\frac{H}{\vartheta}} = \frac{K_H}{p(1+T_{1c}p)}, \, W_{\frac{\theta}{\vartheta}} = \frac{1}{1+T_{1с}p}
\]
\[
W_{\frac{L}{\vartheta}} = \frac{K_L(1+T_Lp)}{p^2(1+T_{1c}p)}
\]
где\\
\[
T_{1c} = \frac{V}{g n_y^\alpha}; \, K_v^{'} = g\cos{\alpha}; \, T_v = -\frac{V n_x^\alpha}{g \cos{\Theta} n_y^\alpha};\, K_H = V \cos{\theta}; \, K_L = g \cos{\theta}^2 ;\, T_L= \frac{V}{g \cos{\theta}} (- \frac{n_x^\alpha}{n_y^\alpha} + \tan{\theta}).
\]
Канал двигателя:\\
\[
W_{\frac{V}{n}} = \frac{K_v}{p};\, W_{\frac{H}{n}} = \frac{K_v \sin{\theta} + (K_H^{'} + K_V^{'} T_{1c} \sin{\theta})p}{p^2(1+T_{1c}p)};
\]
\[
W_{\frac{\theta}{n}} = \frac{K_\theta^{'}}{1+T_{1c}p}; \, W_{\frac{L}{n}} = \frac{K_L^{'}}{p^2} 
\]
где \\
\[
K_v^{'} = g\cos{\alpha}; \, K_\theta^{'} = \frac{\sin{\alpha}}{n_y^\alpha}; \, K_H^{'} = gT_{1c} \cos{\theta}\sin{\alpha};\, K_L^{'} = g\cos{\theta} \cos{\alpha}
\]
Приведем схему полученных передаточных функций пригодных для приближенного исследования продольного возмущенного движения:
схема 1 канал руля высоты:
\begin{figure}[h]
\import{./images/}{fig_5.pdf_tex}
\end{figure}

\subsection{Уравнение бокового возмущенного движения}
Cистема уравнений описывающее боковое возмущенное движение самолета по параметрам $\beta, \, \gamma, \, \omega_x, \, \omega_y$ имеет вид:
\begin{equation}
\begin{cases}
\dot{\beta} = \frac{g}{V} n_z + \omega_y \cos{\alpha} + \omega_x \sin{\alpha} + \frac{g\cos{\vartheta}}{V} \gamma \\
\dot{\omega}_y = \frac{I_{xy}}{I_{y}} \dot{\omega}_x + \bar{M}_y^\beta \beta + \bar{M}_y^{\omega_y} \omega_y + \bar{M}_y^{\omega_x} \omega_x + \bar{M}_y^{\delta_н} \delta_н + \bar{M}_y^{\delta_э} {\delta}_э \\
\dot{\omega}_x = \frac{I_{xy}}{I_{x}} \dot{\omega}_y + \bar{M}_x^\beta \beta + \bar{M}_x^{\omega_y} \omega_y + \bar{M}_x^{\omega_x} \omega_x + \bar{M}_x^{\delta_н} \delta_н + \bar{M}_x^{\delta_э} {\delta}_э \\
\dot{\gamma} = \omega_x - \omega_ y \tan{\vartheta} \\ 
\end{cases}
\end{equation}
В условии $I_{xy} = 0$ в качестве СК выбрать главные ценральные оси инерции.\\
Допущения $\theta = 0, \, \vartheta = \alpha = const$ применив преобразовения Лапласса получаем 
систему $x = Ax + Bu$ бокового движения самолета вокруг центра масс небходимо для исследования ручного режима движения и системы угловой стабилизации самолета. Эта система учитывает взаимосвязь движения рыскания и крена. При полете на малых углах атаки эта взаимосвязь мала и для приближенной оценки динамики управляемого движения можно ипользовать изолированного движения рыскания и крена:\\
\begin{enumerate}
\item Уравнение изолированного движения рыскания. Допущения: $\gamma = 0, \, \omega_x = 0, \, \cos{\alpha} \approx 1,\,\sin{\alpha} = 0$
Тогда система значительно упрощается 
$(pI_n - A)x = B u$
если принебресь поперечной перегрузкой от руля направления, то получим передаточные функции:
\[
W_{\frac{\omega_y}{\delta_Н}}= \frac{\bar{M}_y^{\delta_н}(p - \frac{g}{V} n_z^\beta)}{p^2 + 2h_бp + \omega^2_{0б}}
\]
\[
W_{\frac{\beta}{\delta_н}} = \frac{M_y^{\delta_н}}{p^2 + 2h_бp + \omega^2_{0б}}
\]
где $\omega_{0б}^2 = \frac{g}{V} n_z^\beta \tilde{M}_y^{\omega_y} - \tilde{M}_y^\beta$, $2h_б = -\frac{g}{V}n_z^\beta - \tilde{M}_y^{\omega_y}$
\item Уравнения изолированного движения крена. Допущения: $\beta = 0, \omega_y = 0$.
\[
W_{\frac{\omega_x}{\delta_э}} = \frac{\tilde{M}_x^{\delta_э}}{p - \tilde{M}_x^{\omega_x}}
\]
\[
W_{\frac{\gamma}{\delta_э}} = \frac{\tilde{M}_x^{\delta_э}}{p(p - \tilde{M}_x^{\omega_x})}
\]
\end{enumerate}
\newpage

\section{Лекция 8}
\subsection{Траекторное движение}
Для исследовании самолета по траектории можно исп упрощенную мат. модель. Будм считать $\beta$, $\gamma$ от которых зависит боковая сила параметрами управления изменение параметров $\Psi$ и $V_z$ описывается следующими уравнениями:
\begin{equation}
\begin{cases}
\dot{\Psi} = -\frac{g}{V}(n_{ya} \sin{\gamma_a} + n_z \cos{\beta} \cos{\gamma_a})\\
V_z = -V\sin{\Psi}\\
\dot{z} = V_z\\

\sin{\gamma_a} = \frac{\cos{\vartheta}}{\cos{\beta}} \sin{\gamma} \\
\end{cases}
\end{equation}
линеаризация этих уравнений в окрестности прямолинейного горизонтального движения приводит к следующим уравнениям траекторного движения

\begin{equation}
\begin{cases}
\dot{\Psi} = \frac{-g}{V}(n_y(\alpha)\cos{\alpha}\gamma + n_z)\\
\dot{V}_z = -V \dot{\Psi} = g(n_y^{\alpha}(\alpha) \cos{\alpha\gamma} + n_z)\\
\dot{z} = V_z
\end{cases}
\end{equation}

На основе этих уравний получим передаточные функции (принимая $n_y^{\alpha} \alpha =1 $ 

\[
W_{\frac{V_z}{\gamma}} = \frac{g\cos{\alpha}}{p}, \, W_{\frac{z}{\gamma}} = \frac{g\cos{\alpha}}{p^2}, \, W_{\frac{V_z}{n_z}}= \frac{g}{p}, \, W_{\frac{z}{n_z}} = \frac{g}{p^2}.
\]
Рассматривая изолирваннное движение крена ($\beta = 0$ и $m_z = 0$) получим:

\[
\omega_y + \omega_x \alpha_{гп} + \frac{g}{V}\gamma = 0
\]
\[
\omega_x = \dot{\gamma}
\]
\[
\omega_y(p) = -\gamma(p)(\alpha_{гп}p + \frac{g}{V}n_y^\alpha \alpha_{гп})
\]
Если мы рассматриваем в качестве опорного движение горизонтальный полет, то уравнение переписывается в виде:
\[
\omega_y(p) = -\gamma(p)(\alpha_{гп}p + \frac{g}{V})
\] 
Тогда передаточная функция зависимости угла крена от угла рыскания:
\[
W_{\omega_y/\gamma} = -\alpha_{гп}(p+\frac{g}{V}n_y^\alpha)
\]

\[
W_{\psi/\gamma} = -\frac{\alpha(p + \frac{g}{V} n_y^\alpha)} {p}
\]
Cхема канал элерона:
\begin{figure}[h]
\import{./images/}{fig_6.pdf_tex}
\end{figure}
Cхема канала руля направления:
\begin{figure}[h]
\import{./images/}{fig_7.pdf_tex}
\end{figure}
рассматриваем их изолированное движение.

\subsection{Передаточные функции исполнительных устройств}
\begin{itemize}
\item Привод с жесткой обратной связью
\[
W_{пр} = \frac{1}{T_{пр}^2p^2 + 2 \xi_{пр} Tp + 1}
\]
где $\xi_{пр} = 0.7$, $T_{пр}$ -- постоянная времени, определяется из динамических свойств самолета.

\item ДУС - датчик угловой скорости.
ПФ -- колебательное звено.\\
$\xi = 0.7$, $T \approx 0.01 с$

\item Датчик перегрузок 
ПФ -- колебательное звено.\\
$\xi = 0.7$, $T \approx 0.005 с$
\end{itemize}

\subsection{Астатический интегральный автомат продольного управления}
Отклонение руля:
\begin{equation}
\varphi = K_ш X + K_{\omega_z} \omega_z + K_{ny} \Delta n_y + K_{\int} \int (\Delta n_y + K_x \Delta x) \, dt
\label{eq:stab}
\end{equation}
где $\Delta x = x - x_{бал}$
Рассмотрим режим балансировки самолета в установившемся горизонтальном полете с законом отклонения руля \eqref{eq:stab}.
Балансировочное значение -- $\varphi_{бал} = K_ш X_{бал} + K_{\int} C$\\
Данное свойство закона управления исправляет "лошку" ($X^{бал}(M)$) которе имеет место быть в случае статического АПУ (Обращенное управление).В интегральном АПУ можно обеспечить монотонный характер в зависимости $X^{бал} (M)$ недостающую величину вырабатывает автоматика $K_{\int}C$
Рассмотрим реакцию самолета на ступенчатое отклонение рычага управления $\Delta x = const$. В установившейся фазе переходного процесса должно выполняться:
\[
\Delta n_y + K_x \Delta x = 0
\]
$ X^{n_y} = \frac{\Delta x}{\Delta n_y} = -\frac{1}{K_x} $ - статический коэффициент управляемости является величиной постоянной и не зависит от режима полета (до 21 века). \\
Учитывая выше сказанное нарисуем структурную схему.\\ 
\begin{figure}[H]
\begin{minipage}{\textwidth}
\import{./images/}{fig_8.pdf_tex}
\end{minipage}
\caption{Схема ЭДСУ}
\end{figure}

\begin{figure}[H]
\begin{minipage}{\textwidth}
\import{./images/}{fig_9.pdf_tex}
\end{minipage}
\end{figure}


\[
W_{ос} = -K_{\omega_z} (p+\bar{Y}^\alpha) - n_y^\alpha (\frac{K_{\int}}{p} + K_{n_y})
\approx -K_{\omega_z}\frac{W_{колеб}}{p}
\]
где $2h_1 = \frac{-K_{n_y} n_y^\alpha}{K_{\omega_z}} - \bar{Y}^\alpha$, $\omega_1^2 = -\frac{K_{\int}} {K_{\omega_z}} n_y^\alpha $\\
Найдем передаточную функцию следующего разомкнутого контура:

\[
W_{пр}* = {\frac{\alpha}{\varphi}} = \frac{|\bar{M}_z^\varphi| (p^2 + 2h_1 p + \omega_1^2)} {p(p^2 + 2 hp + \omega_0^2) (p^2 + 2h_{пр}p + \omega_{пр}^2)} 
\]
При выполнения условия $\frac{1}{T} >= 10 \omega_0$ корневой годограф замкнутого контура имеет вид:\\
\begin{figure}[H]
\centering
\import{./images/}{fig_10.pdf_tex}
\end{figure}
Видно что для достаточно больших значений $K_{\omega_z}$ пара комплексно сопряженных корней замыкаются на комплексно сопряженный 0.\\
Изобразим ЛФЧХ разомкнутого контура.
$\omega_1 < \omega_c < \frac{1}{T_{пр}}$

\begin{figure}[H]
\centering
\import{./images/}{fig_11.pdf_tex}
\end{figure}


Полагая что $\omega_с = K_{\omega_z} | \bar{M}_z^\varphi| = \frac{1}{T_{пр}}
(K_{\omega_z})_{гр} = \frac{1}{T_{пр}|\bar{M}_z^\varphi|}$

Для обеспечения запаса амплитуды $A > 12 дБ$ должно выполняться неравенство $\frac{K_{\omega_z}}{(K_{\omega_z})_{гр}} \leq 0.3 \sigma $
Левая граница $\omega_1 \leq 0.15 \frac{1}{T_{пр}}$обеспечивает необходимую длину -20 дБ/дек чем меньше $\omega_1$ тогда увеличивает запас по фазе.
Дадим рекомендации по выбору рациональных значений коэффициентов от которого зависят параметры $h_1, \, \omega_1$ определяющие предельные значения ($K_{\omega_z} \rightarrow\infty$) пары корней.\\
Все сделали для того чтобы выбрать коэффициенты обратной связи. \\ 
Таким образом пологаем, $\omega_1^0 = (0.1 - 0.15) \frac{1}{T_пр}$, $\xi_1^0 = (\frac{h_1}{\omega_1}) = 0.8$.\\
При этом $K_{\omega_z}^0 = \sigma_0 (K_{\omega_z})_{гр}$, $K_{ny}^0 = (2\xi_1^0 \omega_1^0 - \bar{Y}^\alpha) \frac{1}{n_y^\alpha} K_{\omega_z}^0$, $K_{\xi}^0 = \frac{1}{n_y^\alpha} K_{\omega_z}^0$, $\sigma_0 = (0.2 - 0.3)$
\newpage
\section{Лекция 9}
\subsection{Выбор значения $K_x$}
Найдем передаточную функцию системы $\Delta n_y$, $\Delta x$. Привод идеальный.
\begin{equation}
\frac{\Delta n_y}{\Delta x} = K_ш (p + \frac{K_{\int} K_x}{K_ш}) \frac{\bar{M}_z^{\varphi} n_y^\alpha}{p(p^2 +  2 h p + \omega_0^2) - K_{\omega_z} \bar{M}_z^{\varphi} (p^2 + 2h_{1}p + \omega_1^2)}
\end{equation}
Если $p \rightarrow 0$:

\[
 \left\{ \frac{\Delta n_y}{\Delta x} \right\}_{уст}  = -K_{\int}K_x \frac{\bar{M}_z^{\varphi} n_y^\alpha}{K_{\omega_z} \bar{M}_z^{\varphi} \omega_1^2} = -K_x; \: K_{\int} = \frac{\omega_1^2}{n_y^\alpha} K_{\omega_z}
\]
\[
X^{n_y} = - \frac{1}{K_x} = \frac{\Delta X}{(\Delta n_y)_уст}
\]
\[
\frac{\Delta n_y}{\Delta x} = \frac{K_{ш} \bar{M}_z^\varphi n_y^\alpha(p + \lambda_0)}{p^3 + p^2 (2h+\omega_{ср}) + p (\omega_0^2 + \omega_{ср} h_1) - \omega_{ср} \omega_1^2} = \frac{K_{ш} \bar{M}_z^\varphi n_y^\alpha(p + \lambda_0)}{\Delta(p)}
\]
где $\lambda_0 = \frac{K_{\int} K_x}{K_ш}$, $\omega_{ср} = -K_{\omega_z}\bar{M}_z^\varphi$
С другой стороны учитывая пару корней замкнутой системы при $\uparrow \, K_{\omega_z}$. Определяется приближенно из уравнения :
\[
\frac{\Delta n_y}{\Delta x} = \frac{p^2 + 2 h_1 p + \omega_1^2}{\Delta(p)}
\]
где $\Delta(p) = (p^2 + 2 h_1 p + \omega_1^2)(p + \lambda_1)$, $\lambda_1 = \omega_{ср}$. Учитывая $\omega_{ср} = - K_{\omega_z} |\bar{M}_z^\varphi|$, $K_{\omega_z} = \sigma_0 (K_{\omega_z})_{гр}$.\\
\[
(K_{\omega_z})_{гр} = \frac{1}{T_n |\bar{M}_z^\varphi|}
\]
\[
\omega_{ср} = \frac{\sigma_0}{T_n} = \lambda_1
\]
\[
\frac{\Delta n_y}{\Delta x} = \frac{K_{ш} \bar{M}_z^\varphi n_y^\alpha(p + \lambda_0)}{(p^2 + 2h_{1} p + \omega_1^2)(p+\lambda_1)}
\]
Но $\lambda_0$ является источником колебательности следовательно от него нужно избавится, можно обеспечить $\lambda_0 = \lambda_1$.
\[
K_{ш} = \frac{K_{x} K_{\int}}{\sigma_0} T_n = \frac{K_x \omega_1^2}{|\bar{M}_z^\varphi| n_y^\alpha}
\]
С учетом выбора $K_{ш}$:
\begin{equation}
\frac{\Delta n_y}{\Delta x} = -\frac{\omega_1^2 K_x}{p^2 + 2 h_1 p + \omega_1 ^2}
\end{equation}
Эта передаточная функция описывает приближенно систему "самолет + интегральное АПУ".
\subsubsection{Методика синтеза интегрального АПУ}
\begin{enumerate}
\item Задаем $\omega_{{1}_{0}} = (0.1 \div 0.15) \frac{1}{T_n}$, $\xi_1^0 = (0.8 \div 0.9)$, $\sigma_0 = 0.4$.
\item $K_{\omega_z}^0 = \sigma_0 (K_{\omega_z})_{гр} = \frac{\sigma_0}{T_n|\bar{M}_z^\varphi|}$
\item $K_{n_y} = (2\xi_1^0 \omega_1^0 - Y^\alpha)\frac{1}{n_y^\alpha} K_{\omega_z}^0$
\item $K_x^0 = \frac{1}{|X^{n_y}|}$
\item $K_{ш} = \frac{K_x^0 K_{\int}^0}{\sigma^0}T_n$
\item $K_{\int}^0 = \frac{{\omega_1^0}^2 K_{\omega_z}^0}{n_y^\alpha}$
\end{enumerate}
Закон отклонения позволяет:
\begin{enumerate}
\item Обеспечить желаемое изменение положения рычага управления $X_{бал}(H,M)$ в зависимости от параметров движения в разгоне и торможении.
\item Обеспечить желаемое постоянное значение статичеческих характеристик.
\item Обеспечить заданные динамические переходные процессы по $n_y$ (приближенно описывается колебательным звеном).
\end{enumerate}
\subsection{Автомат продольного управления устойчивого самолета}
АПУ -- это СУУ в котором по отрицательной обратной связи по $\omega_z$ вводится обратная связь по $n_y$ или $\omega_z$ и регулировать коэффициент передачи $K_{ш}=\frac{\partial \varphi}{\partial x}$.\\
Дополнительно к летчику:
Руль отклонения по следующему закону:
\[
\Delta \varphi_a  = K_{\omega_z} \Delta \omega_z + K_{n_y} \Delta n_y
\]
\[
\Delta \varphi = K_{ш} + K_{\omega_z}\Delta \omega_z + K_{n_y} \Delta n_y
\]
Будем считать, что перегрузка измеряется аксселерометрами: $\Delta n_y = n_y^\alpha \Delta \alpha$
Оценим влияние обратной связи по $n_y$ на $m_z^{C_y}$ в этом случае имеем приращение коэффицента момента:
\[
\Delta m_z = m_z \varphi K_{ny}\Delta n_y = m_z^\varphi K_{ny} \frac{\Delta Y}{G}
\]
\[
\Delta m_z^{C_y} = m_z^\varphi K_{n_y} \bar{C}_{y_{ГП}} = m_z^\varphi K_{n_y} \frac{qS}{mg}
\]
Следовательно за счет увеличения $m_z^{C_y}$ на $|\Delta m_z^{C_y}|$ происходит увеличение эффективное значение собственных колебаний самолета.
\[
\omega_0^2 = \frac{-qS b_a}{I_{zz}} C_y^\alpha \sigma_n
\]
Функциональная схема для ЭДСУ

\begin{figure}[H]
\centering
\import{./images/}{fig_12.pdf_tex}
\end{figure}
Составим структурную схему с учетом идеальных измерителей:
\begin{figure}[H]
\centering
\import{./images/}{fig_13.pdf_tex}
\end{figure}
\[
\frac{\Delta \alpha}{\Delta \varphi} = \frac{\bar{M}_z^\varphi}{p^2 + 2 h_0 p + \omega_0^2},
\]
\[
\left\{ \frac{\Delta \omega_z}{\Delta \alpha}\right\} = P + \bar{Y}^\alpha,
\]
\[
W_{ос} = -K_{n_y} n_y^\alpha - K_{\omega_z} (P + \bar{Y}^\alpha).
\]
Передаточная функция разомкнутой системы:
\[
W_{раз} = \frac{K_{\omega_z} |\bar{M}_z^{\varphi}|(P + \bar{Y}^\alpha)}{(p^2 + 2h p + \omega_0^2)(T_{пр}^2 p^2 + 2 \xi_{пр} T_{пр} p + 1)}.
\]
Аналогично контуру демпфирования с одним отличием можно регулировать нуль $-\bar{Y}^\alpha$ при условии $10 \omega_0 = \frac{1}{T_{пр}}$.
\begin{figure}[H]
\centering
\import{./images/}{fig_14.pdf_tex}
\end{figure}
Построим АФЧХ:
\begin{figure}[H]
\centering
\import{./images/}{fig_15.pdf_tex}
\end{figure}
На участке $\bar{Y}^\alpha \div \frac{1}{T_{пр}}$ передаточная функция имеет вид:
\[
\frac{K_{\omega_z}|\bar{M}_z^\varphi|}{p},
\]
\[
\omega_{ср} = K_{\omega_z} |\bar{M}|_z^\varphi|,
\]
\[
(K_{\omega_z})_{гр} = \frac{1}{T_{пр} |\bar{M}_z^\varphi|}.
\]
Требуется чтобы длины участка с наклоном $-20 \, \frac{дб}{дек}$ была больше $0.7 \, дек.$ накладывется дополнительное ограничение:

\[
\lg \frac{1}{T_n} - \lg \bar{Y}^{\alpha}_* \geq 0.7;\, \frac{1}{T_n \bar{Y}^{\alpha}_*} \geq 5; \, \bar{Y}^{\alpha}_* \leq \frac{0.2}{T_n}.
\] 

Значение $K_{\omega_z} = 0.25(K_{\omega_z})_{гр}$ определяет запас устойчивости по амплитуде. $T_n \bar{Y}_*^\alpha$ определяет запас по фазе чем меньше произведение, тем больше запас по фазе.
Исходя из условий:
\[
\begin{cases}
\sigma = \frac{K_{\omega_z}}{(K_{\omega_z})_{гр}} \leq 0.25 \\
\bar{Y}_*^\alpha \leq \frac{0.2}{T_{пр}}
\end{cases}.
\]
С учетом этих коэффициентов подчиним выбор коэффициентов $K_{\omega_z}$, $K_{n_y}$ требуется обеспечить желаемую динамику "самолет + АПУ".\\
Установим как зависят корни замкнутой системы пораждаемые от $K_{\omega_z}$, $K_{n_y}$.
\[
\frac{\Delta n_y}{\Delta x} = \frac{\bar{M}_z^\varphi (P + \bar{Y}^\alpha)}{\Delta(p)},
\]
где $\Delta(p) = p^2 + 2(h+0.5 K_{\omega_z} |\bar{M}_z^\varphi|)p + (\omega_0^2 + K_{\omega_z}|\bar{M_z}^\varphi|\bar{Y}^\alpha)$, $\omega_{ср} = K_{\omega_z} |\bar{M}_z^\varphi| \bar{Y}^\alpha$.
АПУ изменяет только параметры знаменателя передаточной функции.
Приравнивая знаменатель нулю получим характеристическое уравнение:
\[
\Delta_{эф}(p)= p^2 - 2 \xi_{эф}\omega_{0 \, эф}p + \omega_{0 эф}^2 = 0, 
\]
\[
\omega_{0 \, эф}^2 = \omega_0^2 + \omega_{ср}\bar{Y}^\alpha_*,
\]
\[
\xi_{эф} = \frac{\omega_0 \xi_с +0.5\omega_{ср}}{\sqrt{\omega_1^2 + \omega_{ср}}} \bar{Y}^{\alpha}_* ,
\]
\[
h = \omega_0 \xi_с,
\]
\[
\omega_{ср}= \frac{\omega_{0^*}^2 - \omega_0^2}{\bar{Y}^{\alpha}_*},
\]
\[
\bar{Y}^{\alpha}_* = \frac{\frac{1}{{\xi_c^*}^2}}{h + 0.5 \omega_{ср}^2- \omega_0^2}{\omega_{ср}},
\]
\[
\omega_{ср}^{потр} = 2(\xi_с^* \omega_0^* - h),
\]
\[
\frac{0.2}{T_n} \geq \bar{Y}^{\alpha}_* = \frac{\omega_0}{\omega_{ср}^{потр}},
\]
\[
K_{\omega_z}^{потр} = \frac{\omega_{ср}^{потр}}{|M_z^{\varphi}|},
\]
\[
K_{\omega_z}^0 = min\{K_{\omega_z}^{потр}; \, 0.25(K_{\omega_z})_{гр}\},
\]
\[
K_{ny}= \lambda^0 K^0_{\omega_z},
\]
\[
\lambda^0 = \frac{1}{n_y^\alpha}(\bar{Y}^{\alpha^{потр}}_* - \bar{Y}^\alpha).
\]
В процедуре синтеза необходимо провести для всех режимов и определить в процессе регулировки парамтры:
\[
K_{\omega_z}(q), K_{ny}^*(q).
\]
Эффективное значение $\xi_с^*$, $\omega_0^*$ вычисляeтся по отдельным формулам:
\[
\omega_{0_{эф}} = \sqrt{\omega_{0^{2}} -\bar{M}_z^{\varphi}} n_y^{\alpha}(\frac{g}{V}),
\]
\[
\xi = \frac{h - 0.5 K_{\omega_z}^* \bar{M}_z^{\varphi}}{\omega_{0_{эф}}}.
\]
Влияние АПУ на статические характеристики $\sigma_n$:
\[
\Delta \sigma_n = \frac{m_z^\varphi q S}{mg}(K_{n_y} + \frac{g}{V}K_{\omega_z}).
\]
2. Для установившейся фазы полета 
\[
\varphi = K_{ш}x + K_{\omega_z} \omega_z \frac{g}{V}\Delta n_y + K_{n_y} \Delta n_y,
\]
\[
X^{n_y} = \frac{\varphi^{n_y}}{K_ш} - \frac{1}{K_{ш}} (K_{\omega_z} \frac{g}{V} + K_{n_y}),
\]
\[
\Delta X^{n_y}= \frac{1}{K_\omega} (\frac{g}{V} K_{\omega_z}+ K_{n_y}).
\]
В состав АПУ также входит автомат регулироваия управления его задача уменьшить заброосы по $n_y$ по режимам полета это осуществляется путем регулирования коэффициента 
\subsection{Синтез статического автомата продольного управления неустойчивого самолет}
В статическом АПУ дополнительное отклонение органа управления осуществляется по закону: 

\[
\Delta \phi = K_{\omega_{z}} \omega_z + K_{n_y} \Delta n_y.
\]
Структурная схема, используемая в синтезе АПУ неустойчивого самолета такая же как и для устойчивого отличие только в передаточной функции 
\[
{\frac{\Delta \alpha}{\Delta \phi} } = \frac{\bar{M}_z^\phi}{(p+\lambda_1)(p+\lambda_2)},
\]
в этом случае передаточная функция размокнутого контура (с обратными свзями)
\[
W_{раз} = \frac{K_{\omega_z} |\bar{M}_z^\phi |(p + \bar{Y}^\alpha_*) }{(p+\lambda_1)(p+\lambda_2)(T_n^2 p^2 + 2 \Xi_n T_n p +1)},
\]

\[
\bar{Y}_*^\alpha = \bar{Y}_*^\alpha + \frac{K_{n_y}}{K_{\omega_z}} n_y^\alpha,
\]
где самолет неустойчив определяется: 
\[
max \lambda_1 (\lambda_1 > \lambda_с),
\]

\[
\hat{\omega} = max(max \: \omega_0^2, max \lambda),
\]

\[
T_n = \frac{1}{10 \hat{\omega}}.
\]
Некоторые соображение по выбору $Y_*^\alpha$ рассмотрим корневой годограф замкнутой системы для двух случаев 
\begin{figure}[h]
\centering
\import{./images}{fig_20.pdf_tex}
\end{figure}

Второй годограф предпочительнее потому что $Y^\alpha$ лежит левее.
Необходимо найти значения $(K_{\omega_z})_{гр}^*$ при которых система будет устойчива.
\begin{enumerate}
\item $\frac{1}{1/T_n \bar{M}_z^\phi}$
\item $\frac{\lambda_2 \lambda_1}{| \bar{M}_z^\phi|\bar{Y}_*^\alpha}$
\end{enumerate}
\[
2.5 (K_{\omega_z})^*_{гр} < K_{\omega_z} < 0.4(K_{\omega_z})^*_{гр}  
\]
\[
\bar{Y}_*^\alpha \le \frac{0.2}{T_n}
\]

Аналогично выбор коэффициентов совпадает с устойчивым самолетом 
\[
(K_{\omega_z})_{потр} = \frac{2 \xi^* \omega^* - (\lambda_1 - \lambda_2)}{|\bar{M}_z^\varphi|},
\]

\[
(Y_*^\alpha) = \frac{(\omega^*)^2 + \lambda_1 \lambda_2}{|\bar{m}_z^\varphi|},
\]

\[
\omega^* =  - \lambda_1 \lambda_2 - \bar{M}_z^\varphi K_{\omega_z} \bar{Y}_*^\alpha, 
\]

\[
\xi^* = \frac{0.5[(\lambda_1 - \lambda_2) - \bar{m}_z^\varphi K_{\omega_z}]}{\sqrt{-\lambda_1\lambda_2 - \bar{M}_z^\varphi K_{\omega_z}\bar{Y}_*^\alpha }}.
\]

Проверка удовлетворения выбранных значений допустимым ограничением

\section{Лекция 10. Семестр 2.}
\subsection{Автоматическое управление угловым движением}
Осуществляется с помощью автопилота, оно основанно на регулированиии углов тангажа, крена, рыскания, по сигналам заданных значений, вырабатываемых или летчиком (с помощью задающих устройств, или в контурах автоматического управления траекторией движения).\\ 
Автоматическое регулирование угла тангажа, АП(автопилота) $\theta$ позволяет реализовать два режима регулирования: стабилизацию и управление траекторией.\\ 
В режиме стабилизации сохраняется исходное угловое положение самолета. Обычно автопилот включают в горизонтальном прямолинейном полете, поэтому режим стабилизации при прочих неизменных полета (главное постоянство скорости) обеспечивает выдерживание прямолинейной траектории. В режиме управления можно реализовать более сложные траектории вводя в автопилот дополнительный командый сигнал $\Delta \theta_{зад}$ перед рычагом управления, можно обеспечить режим спуска или подъема с заданным углом тангажа возможно также приведение самолета из любого положения в горизонтальный полет, задавя $\theta_{зад} = \theta_{гп}$ 

\begin{figure}[h]
\centering
\import{./images}{fig_16.pdf_tex}
\end{figure}

Механизм согласования предназначен для сдвига базы отсчета угла тангажа, измеряемого гировертикалью на величину $\theta - \theta_0$ (момент включения автопилота), что позволяет стабилизированть исходное положение угла танагажа в момент включения автопилота при нулевом значении $\Delta \theta_{зад}$ механизм согласования представляет собой следящую систему, которая находится в замкнутом состоянии когда автопилот не включен в этом случае передаточная функция $\{{\frac{\hat{\theta}}{\theta}}\}= \frac{\frac{k}{p}}{1 + \frac{k}{p}} =\frac{k}{p+1}=\frac{1}{\frac{1}{k}p + 1}$.\\
$\hat{\theta}$ отслеживает $\theta(t)$ тем точнее, чем больше коэфициент $K$. В момент включения автопилота $t_0$ цепь размыкается и на выходе интегратора запоминается сигнал $\theta(t_0)$,а на выходе в системe формируется $\Delta \theta = \theta - \theta(t_0)$. Сигнал ошибки $\delta \theta $ который должен обнулять автопилот имеет вид:
\[
\delta \theta = \theta_{зад} - \theta = \theta(t_0) + \Delta \theta_{зад} - \theta = \Delta \theta_{зад} - \Delta \theta.
\]

\subsection{Синтез АП $\theta$}
Линейные модели в отклонениях от заданого режима полета будет соответствовать следующая структурная схема (измерители считаем идеальными):
\[
\frac{\Delta \omega_z}{\Delta \varphi} = \frac{\bar{M}_z^\varphi (p+\bar{Y}^\alpha)}{p^2 + 2hp + \omega_{0}^2},
\]

\[
W_п(p) = \frac{1}{T_п^2 p^2 + 2 \xi_п T_п p + 1}.
\]
Рассмотрим два способа регулирования: 
Статический и астатический\\
1) Статический автопилот: 
\[
R_{\theta}(p)= K_\theta.
\]
Преобразуем структурную схему к одноконтурнуму виду для этого сигнал $\theta$ поступающий на вход приводов представим в следующем виде: 
\[
\sigma_n = K_{\omega_z}*\omega_z - K_\theta (\Delta \theta_з - \Delta \theta) = K_{\omega_z} - K_{\theta}(\Delta \theta_з - \Delta \theta) = -K_{\theta} \Delta \theta_з + \Delta \theta(p K_{\omega_z} + K_{\theta}) =
\]
\[
= -(K_{\omega_z}p + K_\theta)[\frac{K_\theta}{K_{\omega_z}p + K_{\theta}}\Delta\theta_з - \Delta \theta].
\]
\begin{figure}[h]
\centering
\import{./images}{fig_18.pdf_tex}
\caption{Схема}
\label{fig:18}
\end{figure}

Схема содержит контур с единичной отрицательной связью выбор параметров $K_{\omega_z}, \, K_{\theta}$ проведем исходя из передаточных функций размонкнутого контура
\[
W_{раз} = -(K_{\omega_z} p + K_\vartheta) W_п \frac{1}{p}\left\{{\frac{\Delta \omega_z}{\Delta \varphi}}\right\} = -\frac{\bar{M}_z^\varphi K_{\omega_{z}}(p + \gamma)(p + \bar{Y}^\alpha)}{p(p^2 + 2hp + \omega_0^2)(T_n^2 p^2 + 2 \xi_n T_n p +1)}.
\]
Переходим к выбору параметров $K_{\omega_z}$ и $\gamma$.
Рекомендации по выбору $\gamma$ рассмотрим корневой годограф 

\begin{figure}[H]
\centering
\import{./images/}{fig_19.pdf_tex}
\end{figure}

\textbf{Быстродействие системы} определяется наименьшим корнем характеристического многочлена.
С этой точки зрения лучше выбирать первый случай $\gamma > \bar{Y}^\alpha$.\\

\begin{figure}[H]
\centering
\import{./images/}{fig_21.pdf_tex}
\end{figure}

На участке частоты среза:
\[
W_{раз} = -\frac{\bar{M}_z^\varphi K_{\omega_z}}{p}.
\]
Из модели частоты среза:
\[
\omega_{ср} = -\bar{M}_z^\varphi K_{\omega_z} = -\frac{1}{T_п},
\]
$(K_{\omega_z})_{гр} = \frac{1}{T_n |\bar{M}_z^\varphi|}.$
Исходя из требований запасов:
$K_{\omega_z} = 0.25 (K_{\omega_z})_{гр}.$
Для предварительного выбора значений можно использовать:
\[
K_{\omega_z} = - \frac{0.25 I_z}{{m}_z^\varphi q S b_a T_n},
\]
\[
K_{\vartheta} = - \frac{0.25}{{m}_z^\varphi T_n} \sqrt{\frac{I_z C_y^\alpha |\sigma_\vartheta|}{q S b_a}}.
\]

\subsection{Анализ точности}
Согласно схеме на рис. \ref{fig:18} найдем передаточную функцию замкнутой системы (при условии, что привод идеальный):
\begin{equation}
W_{зам}= \{\frac{\Delta \vartheta}{\Delta \vartheta_{з}}\} =\frac{K_{\vartheta}}{K_{\omega_z}p +K_{\vartheta}}\frac{W_{раз}}{1 + W_{раз}} =\frac{\bar{M}_z^\phi K_{\vartheta}(p + \bar{Y}^\alpha)}{\Delta(p)},  
\label{eq:w_zam_1}
\end{equation}
где $\Delta(p) = p^3 + p^2(2h - \bar{M}_z^\phi K_{\omega_z}) + p(\omega_0^2 -\bar{M}_z^\varphi K_\vartheta - \bar{M}_z^\varphi \bar{Y}^\alpha K_{\omega_z}) -\bar{M}_z^\varphi \bar{Y}^\alpha K_{\vartheta} $.


в устновишвемся режиме $\Delta \theta_з = 1$, найдем чему равняется $\Delta \theta$

\[
\Delta \theta = W_{зам} * \Delta \theta_з = 1* \Delta \theta_з = \Delta \theta_з
\]
При отработке сигнала статической ошибки нет. \\
Оценим влияние возмущений по моменту $\Delta M_z$ на точность стаблизации не сбалансированный момент, оцениваемый величиной $\Delta M_z$ может быть вызван например засчет сброса груза, самое главное что нас интересует, что изменилась центровка самолета. В статическом режиме $\Delta m_z + m_z^\varphi \Delta \varphi$ согласно структурной схеме, $\delta_\theta =\frac{\Delta m_z}{K_\theta m_z^\phi}$ - это есть наша статическая ошибка, с этой точки зрения, рационально выбирать максимальное больше $K_\theta$.
\subsection{Кананическая форма записи передаточной функции АП тангажа}
Представим \ref{eq:w_zam_1} в виде комбинации передаточных звеньев, для этого необходимо найти корнии характеристического уравнения, этот полином 3-ей степени имеет как правило 1 дейстивтельный и два комплексно-сопреженных корня. Для устойчивости системы необходимо $p_1 < 0, \, a<0$, в этом случае $\Delta \theta \Delta \theta_з$ может быть записанна в виде:
\[
W = \frac{T_{1c} p + 1}{(T_{1}p + 1)(T_2^2 p^2 + 2 \xi_2 T_2 p + 1)}
\]
$T_{1c} = 1/Y_*^\alpha$
$T_1 = \frac{-1}{p_1}$
$T_2 = \frac{1}{\sqrt{a^2 + b^2}}$
$\xi = \frac{-a}{\sqrt{a^2 + b^2}}$
Колебательный характер переходного процесса коэффициентом относительного демпфирования $\xi$. Если $\xi < 0.6$, то ее можно увеличить за счет уменьшения $K_\theta$, если $\xi > 0.6$, то ее можно оставить какой есть.
\subsection{Выбор ограничений}
При работе АП \theta параметры самолета такие как $\theta, n_y$ не должны превышать заданных ограничений, кроме того диапазон отклонения стабилизатора (руля) также ограничен. Так как при работе АП $\theta$ обеспечивается $\Delta \theta_з$, то ограничение угла тангажа можно обеспечить за счет ограничение $\Delta \theta_з$ нарисуем структурную схему с учетом таких ограничений:

\fig{21} 13:43

Ограничения для $\varphi$:
\[ \varphi_min - \varphi_{оп} < \Delta \varphi < \varphi_{max} - \varphi{оп}\]
Должна быть $\Delta n_y_{доп}$\\
С учетом демпфера передаточная функция перегрузки: $\frac{\Delta n_y}{\Delta \varphi} = \frac{\bar{M}_z^\phi n_y^\alpha}{p^2 + 2hp + \omega_0^2}$, где $\varphi = \delta \theta (-K_\theta)$
$\Delta n_y = \delta \theta (-K_\theta ) \frac{\bar{M}_z^\phi n_y^\alpha}{p^2 + 2hp + \omega_0^2}W_п$
\[
(\Delta n_y)_{уст} = \delta \theta (- K_\theta)\frac{\bar{M}_z^\phi n_y^\alpha}{(\omega^*)^2} 
\]

\[
n_{y}_{min} < n_y < n_{y}_{min}
\]

\[
n_{y}_{min} -  1< \Delta n_y < n_{y}_{max}
\]
Фото 13 53
\subsection{Регулировка $K_{\omega_z}$ по режимам полета}
$K_{\omega_z}  =\frac{-0.25 I_z}{m_z^\varphi q S b_a T_n} $
$K_{\theta}  = - \frac{-0.25}{m_z^\varphi T_n} \sqrt{\frac{I_z C_y^\alpha |\sigma_n|}{qSb_a}}$
\fig{23} фото 1:58
\section{Астатический АП}

\[
R_{\theta}(p) = K_\theta + \frac{K_u}{p}
\]

\[
R_{\theta}(p) = K_\thta (\frac{T_k p  + 1 }{T_k p})
\]

\[
T_k = \frac{K_\theta}{K_u}
\]
т.к интегральная коррекция вводится для устранения стат. ошибки регулирования, она не должна влиять на частоты близкие к собственной частоте 
\[
\frac{1}{T_k} \leq (0.1 \div 0.2) \omega_{ср} \rarrow \frac{0.25}{T_n}
\]

\[
\frac{K_u}{K_\theta} =  \frac{1}{T_k} \leq (0.025 \div 0.05) \frac{1}{T_n} 
\]

Пример расчета АП тангажа:
Исходные данные для ил 76

\[
I_z = 2.9 * 10^8, m_z^\delta=-1.25, m_z^\omega_z = -12, m_z^{\dot{\alpha}} = -4
\]

\end{document}

